\section{DAG-Traversierung}
Zunächst wird sichergestellt, dass alle Knoten unbesucht sind, indem konsequent erstmal alle Knoten auf unbesucht gesetzt werden, egal ob diese es vorher schon waren oder nicht. Anschließend wird die Methode $visit(v_0)$ aufgerufen, also mit der Wurzel des Baumes. In der Methode visit(v) werdendie folgenden Aktionen nur ausgeführt, wenn v, in dem Fall also der Wurzelknoten, unbesucht ist. Es werden nun alle Knoten durchgegangen, die an den übergebenen Knoten angrenzen, um mit diesen die Methode rekursiv aufzurufen. Danach wird der Wert des übergebenen Knotens ausgegeben und dieser auf besucht gesetzt. \newline
Da immer nur die jeweils angrenzenden Knoten als nächstes besucht werden und kein Knoten mehrere Elternknoten besitzt, wird jeder Knoten nur einmal ausgegeben, va. da dies nur durchgeführt wird, wenn der Knoten unbesucht ist.